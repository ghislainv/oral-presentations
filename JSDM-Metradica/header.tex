%\usepackage{pslatex} % to see correctly a .pdf file on the computer screen
\usepackage{pgf}
\usepackage{color}
\usepackage{graphicx}
\usepackage{amssymb} %symbole de maths
\usepackage{amsmath} %idem
\usepackage[utf8]{inputenc}
%\usepackage{fancyvrb} %give size to verbatim
%\usepackage{hyperref}
\usepackage[english,francais]{babel}
\definecolor{vertmoyen}{RGB}{51,110,23} % vert moyen
\definecolor{blueFRB}{HTML}{31859c}
\usecolortheme[named=blueFRB]{structure}
\usepackage{tabularx} % varier la largeur du tableau
\usepackage{layout}
\usepackage{longtable}
\setlength{\LTleft}{-5cm plus 1 fill}
\setlength{\LTright}{-5cm plus 1 fill}
\usepackage{booktabs}
\usepackage{arydshln} %% dashlines for tabular
\newcommand{\logit}{\text{logit}}
\newcommand{\bs}[1]{\boldsymbol{#1}}
\newcommand{\R}{\textnormal{\sffamily\bfseries R}}
\newcommand{\pkg}[1]{{\fontseries{b}\selectfont #1}}
\newcolumntype{C}[1]{>{\centering\arraybackslash}m{#1}}
%% Natbib is a popular style for formatting references.
%\usepackage{natbib} %doesn't work with beamer

\title[JSDM-Metradica]{Assessing tree species vulnerability to climate change in French Guiana using joint species distribution models}
%\subtitle{} 

\date{}

% Theme
% \usetheme{AnnArbor}
% \usetheme{Dresden}
% \usetheme{Copenhagen}
% \usetheme{Frankfurt}
% \usetheme{Berlin}
% \usetheme{Madrid}
% \usetheme{Montpellier}
% \usetheme{Singapore}
% \usetheme{Antibes}
\useinnertheme{rounded} %% bullets
\useoutertheme[subsection=false]{miniframes}
\setbeamertemplate{footline}[frame number]
\setbeamertemplate{frametitle}{%
    \usebeamerfont{frametitle}\insertframetitle%
    \vphantom{g} % To avoid fluctuations per frame
    \par
    \centering \includegraphics[width=\textwidth]{figs/Barre_couleur}
}

% Ignore ignorenonframetext class option in default template
\makeatletter
\beamer@ignorenonframefalse
\makeatother

% Logo
\newif\ifplacelogo % create a new conditional
\logo{\ifplacelogo\includegraphics[width=0.5\textwidth]{figs/partners_logos}\fi} % replace with your own command

%Call table of contents at the beginning of each section
\AtBeginSection[]{
\placelogotrue
  \begin{frame}
    \frametitle{Plan}
    \begin{columns}[c]
      \begin{column}{0.5\textwidth}
        \tableofcontents[sections=1,currentsection]
        \vspace{0.5cm}
        \tableofcontents[sections=2,currentsection]
      \end{column}
      \begin{column}{0.5\textwidth}
        \tableofcontents[sections=3,currentsection]
        \vspace{0.5cm}
        \tableofcontents[sections=4,currentsection]
      \end{column}
    \end{columns}
  \end{frame}
\placelogofalse
}

\AtBeginSubsection[]{}

% \AtBeginSubsection[]{
% \placelogotrue
%   \begin{frame}
%     \frametitle{Plan}
%     \begin{columns}[c]
%       \begin{column}{0.5\textwidth}
%         \tableofcontents[sections={1-2},currentsection,currentsubsection]
%       \end{column}
%       \begin{column}{0.5\textwidth}
%         \tableofcontents[sections={3-4},currentsection,currentsubsection]
%       \end{column}
%     \end{columns}
%   \end{frame}
% \placelogofalse  
% }

% Two-columns slides
\def\bcols{\begin{columns}}
\def\bcol{\begin{column}}
\def\ecol{\end{column}}
\def\ecols{\end{columns}}
