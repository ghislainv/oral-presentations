% Created 2022-11-24 jeu. 17:54
% Intended LaTeX compiler: pdflatex
\documentclass[10pt,table,dvipsnames,compress]{beamer}
\usepackage[utf8]{inputenc}
\usepackage[T1]{fontenc}
\usepackage{graphicx}
\usepackage{longtable}
\usepackage{wrapfig}
\usepackage{rotating}
\usepackage[normalem]{ulem}
\usepackage{amsmath}
\usepackage{amssymb}
\usepackage{capt-of}
\usepackage{hyperref}
\usetheme{default}
\useinnertheme{rounded}
\useoutertheme[subsection=false]{miniframes}
\date{}
\title{Cartes de carbone forestier pour le suivi et la conservation des forêts en Nouvelle-Calédonie}
\title[CARBOFORCAL]{Cartes de carbone forestier pour le suivi et la conservation des forêts en Nouvelle-Calédonie}
\usepackage{lmodern}
\usepackage{pgf}
\usepackage{color}
\usepackage[english,french]{babel}
\definecolor{vertmoyen}{RGB}{51,110,23} % vert moyen
\definecolor{blueFRB}{HTML}{31859c}
\usecolortheme[named=blueFRB]{structure}
\usepackage{tabularx} % varier la largeur du tableau
\usepackage{layout}
\setlength{\LTleft}{-5cm plus 1 fill}
\setlength{\LTright}{-5cm plus 1 fill}
\usepackage{booktabs}
\usepackage{arydshln} %% dashlines for tabular
\newcommand{\logit}{\text{logit}}
\newcommand{\bs}[1]{\boldsymbol{#1}}
\newcommand{\R}{\textnormal{\sffamily\bfseries R}}
\newcommand{\pkg}[1]{{\fontseries{b}\selectfont #1}}
\newcolumntype{C}[1]{>{\centering\arraybackslash}m{#1}}

\setbeamertemplate{footline}[frame number]
\setbeamertemplate{frametitle}{%
\usebeamerfont{frametitle}\insertframetitle%
\vphantom{g} % To avoid fluctuations per frame
\par
\centering \includegraphics[width=\textwidth]{figs/Barre_couleur}
}
\beamertemplatenavigationsymbolsempty

% Logo
\newif\ifplacelogo % create a new conditional
\logo{\ifplacelogo\includegraphics[width=0.4\textwidth]{figs/partners_logos}\fi}

%Call table of contents at the beginning of each section
\AtBeginSection[]{
\placelogotrue
\begin{frame}
\frametitle{Plan}
\begin{columns}[c]
\begin{column}{0.5\textwidth}
\tableofcontents[sections=1,currentsection]
\vspace{0.5cm}
\tableofcontents[sections=2,currentsection]
\end{column}
\begin{column}{0.5\textwidth}
\tableofcontents[sections=3,currentsection]
\vspace{0.5cm}
\tableofcontents[sections=4,currentsection]
\end{column}
\end{columns}
\end{frame}
\placelogofalse
}

\AtBeginSubsection[]{}

\hypersetup{
colorlinks=true,
linkcolor=Black,
filecolor=Maroon,
citecolor=Blue,
urlcolor=Maroon}

% Disable monospaced font for URLs
\urlstyle{same}

\hypersetup{
 pdfauthor={Ghislain Vieilledent},
 pdftitle={Cartes de carbone forestier pour le suivi et la conservation des forêts en Nouvelle-Calédonie},
 pdfkeywords={},
 pdfsubject={},
 pdfcreator={Emacs 27.1 (Org mode 9.5.3)}, 
 pdflang={English}}
\begin{document}


% Title page
{
  \setbeamertemplate{navigation symbols}{}
  \begin{frame}[plain, noframenumbering]
  \begin{center}
  \small{\textbf{CARBOFORCAL -- Nouméa -- Vendredi 25 Novembre 2022}}
  \end{center}
  \vspace{-0.5cm}
  \titlepage % Presentation first page
  \vspace{-3cm}
  \begin{center}
    \includegraphics[width=\textwidth]{figs/Barre_couleur}
    
    \vspace{0.25cm}
    
    \includegraphics[width=10cm]{figs/Banniere}
    
    \small{Ghislain VIEILLEDENT$^{1, 2}$\hspace{0.25cm}Thomas IBANEZ$^{1, 2}$\hspace{0.25cm}CARBOFORCAL$^{2}$}
      
    \vspace{0.25cm}
    
    {\scriptsize
      \begin{tabular}{l}
        $[1]$ \textbf{Cirad} UMR AMAP, $[2]$ \textbf{CARBOFORCAL} UMR AMAP, Oeil, IAC
      \end{tabular}
    }

    \vspace{0.25cm}
    
    \includegraphics[width=0.5\textwidth]{figs/partners_logos}
    
  \end{center}
  \end{frame}
}

% %%%%%%%%%%%%%%%%%%%%%%%%%%%%%%%%%%%%%%%%%%%%%%%%%%%%%%%%%%%%%%%%

\placelogotrue
\begin{frame}
  \frametitle{Plan}
  \begin{columns}[c]
    \begin{column}{0.5\textwidth}
      \tableofcontents[sections=1]
      \vspace{0.5cm}
      \tableofcontents[sections=2]
    \end{column}
    \begin{column}{0.5\textwidth}
        \tableofcontents[sections=3]
        \vspace{0.5cm}
        \tableofcontents[sections=4]
    \end{column}
  \end{columns}
\end{frame}
\placelogofalse

\section{Introduction}
\label{sec:orgbac7b38}

\subsection{Contexte}
\label{sec:orgfefc841}

\begin{frame}[label={sec:org7bfa81a}]{Changements climatiques dans le Pacifique}
\begin{itemize}
\item Changement climatique lié aux émissions de GES dans l'atmosphère.
\item Les communauté du Pacifique seront les premières impactées (montée des eaux).
\item Impact sur la biodiversité (Pouteau et al.: 52-84\% des espèces d'arbres perdront > 50\% de leur habitat)
\end{itemize}

\begin{figure}[htbp]
\centering
\includegraphics[width=\textwidth]{figs/Dutheil2021_cc.jpg}
\caption{Prévisions de changement de précipitations en NC. Dutheil et al. 2021.}
\end{figure}
\end{frame}

\begin{frame}[label={sec:orgd5b5b80}]{Déforestation en Nouvelle-Calédonie}
\begin{itemize}
\item Déforestation et dégradation des forêts: 10--20\% des émissions de GES dans l'atmospère
\item Couvert forestier: 7745 km\textsuperscript{2} en 2020 (\(\sim\) 46 \% du territoire).
\item Inclus: forêt humide, sèche, mangroves, plantations, maquis arbustif (cf. définition FAO: H > 5 m, > 10\% de couvert arborée).
\item Déforestation: 31 km\textsuperscript{2}/an sur la période 2010-2020.
\end{itemize}

\begin{figure}[htbp]
\centering
\includegraphics[width=0.8\textwidth]{figs/deforestation-NC.jpg}
\caption{Déforestation 2000--2010--2020 en Nouvelle-Calédonie. \url{https://forestatrisk.cirad.fr/newcal}}
\end{figure}
\end{frame}

\begin{frame}[label={sec:org0a20960}]{Stock et émissions de carbone}
\begin{columns}
\begin{column}{0.6\columnwidth}
\begin{itemize}
\item Pas d'estimation précise des émissions de carbone (et CO\textsubscript{2}) associées à la déforestation.
\item Manque d'information sur:
\begin{itemize}
\item Les stocks de carbone forestier à l'échelle de la Nouvelle-Calédonie.
\item La variation de ces stocks dans l'espace (structure et composition forestière variables).
\end{itemize}
\item Estimation ponctuelle (inventaires) des stocks de carbone en Nouvelle-Calédonie (150 t/ha \textpm{}42, Blanchard et al. 2016).
\item Cartes globales des stocks de carbone mais n'incluant pas de données de terrain en Nouvelle-Calédonie.
\end{itemize}
\end{column}

\begin{column}{0.4\columnwidth}
\begin{figure}[htbp]
\centering
\includegraphics[width=0.85\textwidth]{figs/Cstock_Mada.jpg}
\caption{Carte des stocks de carbone forestier à Madagascar}
\end{figure}
\end{column}
\end{columns}
\end{frame}

\subsection{Objectifs}
\label{sec:orgd5c91f2}

\begin{frame}[label={sec:org4adc310}]{Objectifs}
\begin{itemize}
\item Elaboration de cartes de carbone forestier.
\item A haute résolution (\(\leq\) 100 m).
\item Avec suivi périodique dans le temps (annuel ou pluriannuel).
\end{itemize}

\begin{figure}[htbp]
\centering
\includegraphics[width=0.7\textwidth]{figs/Baccini2017_Cchange.jpg}
\caption{Carte de changement des stocks de carbone (2004--2014). Baccini et al. 2017.}
\end{figure}
\end{frame}

\subsection{Intérêts}
\label{sec:orgf5f5751}

\begin{frame}[label={sec:orga299f4d}]{Intérêts des cartes de carbone forestier}
\begin{enumerate}
\item \alert{SCIENCE} \\
Rôle des forêts tropicales de Nouvelle-Calédonie dans le cycle du carbone.
\item \alert{TERRITOIRE} \\
Suivi de l'évolution du couvert forestier.
\item \alert{ECONOMIE} \\
Participation au mécanisme REDD+ et financement de projets de conservation des forêts.
\end{enumerate}
\end{frame}

\begin{frame}[label={sec:orge912763}]{Intérêts pour la SCIENCE}
Rôle des forêts tropicales de NC dans le cycle du carbone:

\begin{itemize}
\item Effets des facteurs environnementaux sur le stockage du carbone.
\item Estimation précise des stocks de carbone forestier en Nouvelle-Calédonie.
\item Estimation des émissions liées à la déforestation en Nouvelle-Calédonie.
\end{itemize}

\begin{center}
\includegraphics[width=0.6\textwidth]{figs/carbon_cycle.png}
\end{center}
\end{frame}

\begin{frame}[label={sec:orgfa80b66}]{Intérêts pour le TERRITOIRE}
\begin{itemize}
\item Suivi direct du changement de couvert forestier via un indice de carbone:
\begin{itemize}
\item Déforestation: perte totale de C.
\item Dégradation: diminution des stocks de C, mortalité des arbres, perturbations.
\item Séquestration: augmentation des stocks de C, croissance/recrutement des arbres + reforestation.
\end{itemize}
\item Identification des zones de hotspot de déforestation et des zones de conservation prioritaires.
\item Suivi de la biodiversité et des services écosystémiques (disponibilité en eau, lutte contre l'érosion) associés à la forêt.
\end{itemize}
\end{frame}

\begin{frame}[label={sec:orgf18a307}]{Intérêts ECONOMIQUES}
\begin{itemize}
\item Participation au mécanisme REDD+: Reducing Emissions from Deforestation and forest Degradation.
\item Les tonnes de CO\textsubscript{2} non-émises (évitées) peuvent \alert{potentiellement} être créditées.
\item Estimation à 6 \texteuro{}/t de CO\textsubscript{2} pour les projets forestiers sur le marché volontaire (en hausse constante).
\item Nouvelle-Calédonie, 3100 ha/an x 150 t/ha x 44/12 x 6 \texteuro{}/t = 10 M\texteuro{}/an.
\end{itemize}

\begin{center}
\includegraphics[width=0.4\textwidth]{figs/REDD.jpg}
\end{center}
\end{frame}

\begin{frame}[label={sec:org0e1055a}]{Intérêts ECONOMIQUES}
\begin{itemize}
\item \alert{Pourrait} permettre d'attirer les investisseurs pour la conservation des forêts en Nouvelle-Calédonie.
\item Compensation des émissions des activités humaines (mines et agriculture).
\item Participation supplémentaire et affichage de la Nouvelle-Calédonie à la lutte contre les émissions de CO\textsubscript{2} et le changement climatique.
\end{itemize}

\begin{center}
\includegraphics[height=0.3\textheight]{figs/mines.jpg}
\includegraphics[height=0.3\textheight]{figs/agriculture.jpg}
\end{center}
\end{frame}


\section{Méthode}
\label{sec:org3675ebd}

\subsection{Données de terrain}
\label{sec:orgbb12497}

\begin{frame}[label={sec:orgcfdaa1f}]{Données de terrain}
\begin{center}
\includegraphics[width=0.6\textwidth]{figs/ncpippn-plots.png}
\end{center}
\end{frame}

\subsection{Survol drone LiDAR}
\label{sec:orgc9ef24d}

\begin{frame}[label={sec:org330fd7c}]{Survol drone LiDAR}
Relation entre le carbone estimé sur le terrain et les métriques de hauteur de canopée dérivée du LiDAR

\begin{center}
\includegraphics[width=0.3\textwidth]{figs/lidar-drone.jpg}
\includegraphics[width=0.4\textwidth]{figs/map-lidar.png}
\end{center}
\end{frame}

\subsection{Cartographie via images Sentinel-2}
\label{sec:org33f8373}

\begin{frame}[label={sec:org11a03be}]{Cartographie via images Sentinel-2}
\begin{figure}[htbp]
\centering
\includegraphics[width=0.5\textwidth]{figs/newcal-S2.png}
\caption{Mosaïque d'images Sentinel-2 couvrant la Nouvelle-Calédonie (2020).}
\end{figure}
\end{frame}

\subsection{Trois échelles}
\label{sec:org1d7cba3}

\section{Montage de projet}
\label{sec:orga7e03ee}

\subsection{Institutions partenaires}
\label{sec:orge9436d5}

\begin{frame}[label={sec:orge9df2b0}]{Institution partenaires}
\begin{itemize}
\item UMR AMAP:
\begin{itemize}
\item coordinateurs: Ghislain Vieilledent et Thomas Ibanez.
\item participants: équipe d'une dizaine de personnes, spécialités: \\
botanique, LiDAR, télédétection, modélisation statistique, cartographie.
\item localisation: Nouméa et Montpellier.
\end{itemize}
\item Oeil:
\begin{itemize}
\item participants: Fabien Albouy et Adrien Bertaud.
\item spécialités: étude environnementales, géomatique, communication.
\end{itemize}
\item IAC:
\begin{itemize}
\item participante: Audrey Léopold.
\item spécialités: services écosystémiques, foresterie et biogéochimie.
\end{itemize}
\end{itemize}

\begin{center}
\includegraphics[width=0.4\textwidth]{figs/partners_logos.png}
\end{center}
\end{frame}

\subsection{Calendrier}
\label{sec:orgfb28043}

\begin{frame}[label={sec:org2acca48}]{Calendrier}
\begin{itemize}
\item Projet de 3 ans.
\item 4 workpackages: Inventaires, LiDAR, Cartographie, Coordination.
\end{itemize}

\begin{center}
\includegraphics[width=0.9\textwidth]{tabs/diagramme-gantt.png}
\end{center}
\end{frame}

\subsection{Financement}
\label{sec:org8193605}

\begin{frame}[label={sec:org9ddd122}]{Budget}
\begin{itemize}
\item Budget total: \textasciitilde{}525,000 euros
\item Frais de personnel (15 personnes, 53 mois ETP): 375,000 euros.
\item Frais de fonctionnement (matériels, missions): 150,000 euros.
\end{itemize}

\begin{center}
\includegraphics[width=0.7\textwidth]{tabs/budget_Projet.png}
\end{center}
\end{frame}

\begin{frame}[label={sec:orgc99c5fe}]{Financement}
\begin{itemize}
\item Financement propre: 200,000 euros (> 50\% des frais de personnel).
\item Recherche de financement: 325,000 euros.
\item Co-financements envisageables.
\end{itemize}

\begin{center}
\includegraphics[width=0.7\textwidth]{tabs/budget_Projet.png}
\end{center}
\end{frame}


% %%%%%%%%%%%%%%%%%%%%%%%%%%%%%%%%%%%%%%%%%%%%%%%%%%%%%%%%%%

{
  % Use background image
  \usebackgroundtemplate{%
    \includegraphics[keepaspectratio=true, height=\paperheight]{figs/Canopy-NC}
  }
  \setbeamertemplate{navigation symbols}{}
  % Remove shadow from block
  \setbeamertemplate{blocks}[rounded][shadow=false]
  \begin{frame}[plain]
  	\vspace*{\stretch{100}} 
    \begin{block}{}
      \begin{center}
        \ldots~Merci pour votre attention~\ldots \\
        \url{https://ecology.ghislainv.fr/presentations} \\
        \includegraphics[width=0.45\textwidth]{figs/partners_logos}
      \end{center}
    \end{block}
  \end{frame}
}
\end{document}