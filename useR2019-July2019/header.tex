%\usepackage{pslatex} % to see correctly a .pdf file on the computer screen
\usepackage{pgf}
\usepackage{color}
\usepackage{graphicx}
\usepackage{amssymb} %symbole de maths
\usepackage{amsmath} %idem
\usepackage{upgreek} %idem
\usepackage[utf8]{inputenc}
\usepackage{listings} % code display
\usepackage{marvosym} %\MVRightarrow
%\usepackage{fancyvrb} %give size to verbatim
%\usepackage{hyperref}
\usepackage[english,francais]{babel}
\definecolor{vertmoyen}{RGB}{51,110,23} % vert moyen
\definecolor{blueFRB}{HTML}{31859c}
\usecolortheme[named=blueFRB]{structure}
\usepackage{tabularx} % varier la largeur du tableau
\usepackage{layout}
\usepackage{longtable}
\setlength{\LTleft}{-5cm plus 1 fill}
\setlength{\LTright}{-5cm plus 1 fill}
\usepackage{booktabs}
\usepackage{arydshln} %% dashlines for tabular
\newcommand{\logit}{\text{logit}}
\newcommand{\bs}[1]{\boldsymbol{#1}}
\newcommand{\R}{\textnormal{\sffamily\bfseries R}}
\newcommand{\pkg}[1]{{\fontseries{b}\selectfont #1}}
\newcolumntype{C}[1]{>{\centering\arraybackslash}m{#1}}
%% Natbib is a popular style for formatting references.
%\usepackage{natbib} %doesn't work with beamer

\title[Using Rcpp* packages for easy and fast Gibbs sampling MCMC from within R]{\textbf{Using Rcpp* packages for easy and fast Gibbs sampling MCMC from within R}}
%\subtitle{} 

\date{}

% Theme
% \usetheme{AnnArbor}
% \usetheme{Dresden}
 \usetheme{Copenhagen}
% \usetheme{Frankfurt}
% \usetheme{Berlin}
% \usetheme{Madrid}
%\usetheme{Montpellier}
% \usetheme{Singapore}
% \usetheme{Antibes}
\useinnertheme{rounded} %% bullets
\setbeamertemplate{footline}[frame number]
% \setbeamertemplate{frametitle}{%
%     \usebeamerfont{frametitle}\insertframetitle%
%     \vphantom{g} % To avoid fluctuations per frame
%     \par
%     \centering \includegraphics[width=\textwidth]{figs/Barre_couleur}
% }

% Ignore ignorenonframetext class option in default template
\makeatletter
\beamer@ignorenonframefalse
\makeatother

% Logo
\newif\ifplacelogo % create a new conditional
\logo{\ifplacelogo    \begin{tabular}{C{2cm}C{2cm}C{5cm}}
        \includegraphics[height=0.7cm]{figs/Logo-Cirad.png} &
        \includegraphics[height=1cm, width=1cm]{figs/logo-AMAP.png} 
        \includegraphics[height=0.8cm]{figs/titre-long.png} &
        ~
      \end{tabular}\fi} % replace with your own command

%Call table of contents at the beginning of each section
\AtBeginSection[]{
\placelogotrue
  \begin{frame}
    \frametitle{Plan}
    \tableofcontents[currentsection, hideothersubsections, sectionstyle=show/hide]
  \end{frame}
\placelogofalse
}

\AtBeginSubsection[]{}

% \AtBeginSubsection[]{
% \placelogotrue
%   \begin{frame}
%     \frametitle{Plan}
%     \begin{columns}[c]
%       \begin{column}{0.5\textwidth}
%         \tableofcontents[sections={1-2},currentsection,currentsubsection]
%       \end{column}
%       \begin{column}{0.5\textwidth}
%         \tableofcontents[sections={3-4},currentsection,currentsubsection]
%       \end{column}
%     \end{columns}
%   \end{frame}
% \placelogofalse
% }


% Two-columns slides
\def\bcols{\begin{columns}}
\def\bcol{\begin{column}}
\def\ecol{\end{column}}
\def\ecols{\end{columns}}

% C++ code display
\definecolor{Darkgreen}{rgb}{0,0.4,0}
\definecolor{listinggray}{gray}{0.9}
\definecolor{lbcolor}{rgb}{0.9,0.9,0.9}
\lstset{
backgroundcolor=\color{lbcolor},
    tabsize=4,    
%   rulecolor=,
    language=[GNU]C++,
        basicstyle=\footnotesize,
        upquote=true,
        aboveskip={0.0\baselineskip},
        columns=fixed,
        showstringspaces=false,
        extendedchars=false,
        breaklines=true,
        prebreak = \raisebox{0ex}[0ex][0ex]{\ensuremath{\hookleftarrow}},
        frame=single,
        numbers=left,
        showtabs=false,
        showspaces=false,
        showstringspaces=false,
        identifierstyle=\ttfamily,
        keywordstyle=\color[rgb]{0,0,1},
        commentstyle=\color[rgb]{0.026,0.112,0.095},
        stringstyle=\color[rgb]{0.627,0.126,0.941},
        numberstyle=\color[rgb]{0.205, 0.142, 0.73},
%        \lstdefinestyle{C++}{language=C++,style=numbers}’.
}
\lstset{
    backgroundcolor=\color{lbcolor},
    tabsize=4,
  language=C++,
  captionpos=b,
  tabsize=3,
  frame=lines,
  numbers=left,
  numberstyle=\tiny,
  numbersep=5pt,
  breaklines=true,
  showstringspaces=false,
  basicstyle=\footnotesize,
%  identifierstyle=\color{magenta},
  keywordstyle=\color[rgb]{0,0,1},
  commentstyle=\color{Darkgreen},
  stringstyle=\color{red}
  }